\subsection{Overview}
The basic architecture for the system is a Client-Server setup. The server will handle persistent data and contain a small subsystem for manipulating data, we will for now not describe the server or the communication between the server and client greater detail and instead focus on the client.\\

The Client-system will be made with a Model-View-Controller (MVC) architecture. The system should have different views (Log in, MonthView... etc.) which would require different event-handling. The MVC allows this because it makes it easy to do runtime changes of views and controllers. It also results in high cohesion, in that the control objects and the view objects are separated in accordance to their respective responsibilities. Both view and control objects can be separated into small classes that only handles smaller tasks. However creating the MVC-architecture requires extra code. The main architecture can be seen below (it is possible to zoom in), we have not included all views and control objects to keep things a bit clean.

\paragraph{Architecture}
\begin{center}
\includegraphics[scale=.4]{sections/Architecture.pdf}
\end{center}
\pagebreak

\subsection{Subsystem decomposition}
The MVC architecture gives us three distinct types of subsystems, a model that handles date, a view that handles UI, and a Control object that handles event-flow and manipulation of data. The model subsystem should function independently from the rest of the system, therefore we used a facade design pattern to hide the internal implementation of the model from the rest of the system. In addition, the model communicates with the different views through a Observer/Observable-pattern, which mean the model isn't coupled to the Views.\\

The rest of the system is divided into View and Control objects. Each View requires a Control object, that handles the event-flow of the View. Each Control object has a reference to the ViewHandler, which allows them to create a new View when necessary. \\

At this point we have identified three types of views, a NavigatorView that shows calendar navigation, a AppointmentView that shows an appointment, and a login view that allows a guest to log-in and/or create a user. Each view have one of several control objects, which handles the events caused by the user. There might be several View objects for each Control object fx. the NavigateCalender control object could be responsible for MonthView, WeekView, and DayView under the common name NavigatorView. However, the opposite is not true. One view object cannot be controlled by more than one control object. 

\subsection{Hardware/software mapping}
Since the system uses a Client-Server architecture, our software will have to run on two different pieces of hardware. First off, we have the clients - these are not required to be computation heavy machines. They are required to run Windows, and should be able to connect to the Internet. 
The server on the other hand, should be fairly quick machines, and have a pretty fast internet connection. The server will also run windows, and should be able to run operations offline such as matching appointments or taking backup copies. How this will be implemented in practice is yet to be determined. 


The different responsibilities of the server and client are as follows:
The Server:
\begin{itemize}
	\item Storing all data
	\item Backing up data
	\item Running our LuckyMatchFinder algorithm
	\item Authorizing the clients requests
\end{itemize}
The Client:
\begin{itemize}
	\item Keeping the local users credentials
	\item Synchronizing the local calendar with the server
	\item Saving changes when in a offline state, and later sending them to the server
	\item Providing the user with an easy-to-access UI
\end{itemize}

\begin{center}
\includegraphics[scale=.8]{sections/hardwaremapping.pdf}
\end{center}

On the server side we will use the subsystem Datahandler to stay connected to the clients through a http-connection. The Client will use the model-subsystem to stay connected to the server. The clients will be logged on locally with after an authorization from the server.
The servers DataHandler will function as a wrapper class around a SQL-database, through which it will load and save data after request from the client or the LuckyMatchFinder algorithm. 

If a client for some reason should lose connection to the server, it will be able to save the changes locally instead, and will upon reconnection to the server, synchronize these changes with it. We might want to always save changes locally, but have yet to figure out the exact pros and cons, but this isn't plan yet. The clients model-subsystem uses a bridge pattern to apply the different data handling interfaces, which are assigned though a Strategy pattern which uses an Abstract Factory pattern to create the different data manager classes. The Client will also be able to synchronize with Google Calendar as a secondary server - that is, it will not replace our server, but rather work as a supplement for the user. 


\subsection{Persistent data management}
We have identified the following persistent objects; APPOINTMENTS, LUCKYAPPOINTMENTS, USERS and NOFICATIONS. They are all saved by the server subsystem in a relational Database. When the Client starts up it synchronizes the APPOINTMENTS, LUCKYAPPOINTMENTS and NOTIFICATIONS with the server, so that the Client subsystem is up to date with the changes that have been made by other users and the LuckyMatchFinder. If local changes have not been saved to the server (probably due to loss of network connection) since the program was last used by the logged on User.

We considered saving persistent boundary objects locally, such as user preferences and last shown view, but we decided against it, since the client is supposed to be simple and intuitive, with ease of use and accessibility as declared design goals, making these boundary objects persistent, should not be necessary. 


\subsection{Access control and security}
In our system all users have the same access to the all actions related to their appointment. Users are defined as authenticated users, so the user has already been through the authentication process and can only access its own appointments. All users that have been added normally to an appointment are participants and all have access to edit the appointment. The users added to the the appointment to the LuckyMatchFinder by making a LuckyAppointment are defined as LuckyParticipants and do not have any possibilities to edit the Appointment.\\

We will not encrypt the data transfer of the appointments to between the server and the client, but we wish to encrypt the transfer of the user names by with the password as the key, so that only the user is allowed to his own username and the server then stores the usernames and links them to the relevant appointments, so the authenticated user can access them. \\

No one is allowed to delete an appointment, an appointment is deleted when the last participants leaves. 
However a participant may remove other participants from an appointment, which in reality gives the power of delete appointment.
Access matrix for Participants and LuckyParticipants rights for appointment-functionality.
{\tabulinesep=1.2mm
\begin{tabu}{ | p{3cm} | p{6cm} | p{6cm} |}
    \hline
     	 			        & 		Participant         &		LuckyParticipant \\ \hline
    EditAppointment         &       +					&		-          \\\hline
    LeaveAppointment        &       +					&		+      \\ \hline
    RemoveParticipants		&		+					&		- 		\\\hline
    DeleteAppointment       &       -					&		-        \\  \hline
\end{tabu}
}

+ access matrix??
 
\subsection{Global software control flow}
The Client will generally be using an event-driven control flow. There will be a main-loop listening to for GUI-events and when an event happens it is handled by the relevant controller-class. We still might want to have several threads running, e.g. the ConnectionHandler. Similarly the LuckyMatchFinder running on the server will probably be pretty procedural as it is run x times a day and never even waits events.


\subsection{Boundary conditions}
We have identified the following boundary conditions on the client and the server:


\begin{itemize}
	\item Configuration:
	\begin{enumerate}
		\item We have found no undefined changes to persistent data in our use cases. All changes to appointments has been accounted for by either the user or the server
	\end{enumerate}
\end{itemize}

+ inkluder i use-cases??
